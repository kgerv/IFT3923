\documentclass{article}

\usepackage[french]{babel} 
\usepackage[T1]{fontenc}
\usepackage{lmodern}
\usepackage[utf8]{inputenc}

\usepackage{float}
\usepackage{mathtools}
\DeclarePairedDelimiter{\ceil}{\lceil}{\rceil}
\usepackage{fancyhdr}
\usepackage{extramarks}
\usepackage{amsmath}
\usepackage{amsthm}
\usepackage{amsfonts}
\usepackage{amssymb}
\usepackage{tikz}
\usepackage{graphicx}
\usepackage{pdfpages}
\usepackage[makeroom]{cancel}
\usepackage{karnaugh-map}

%
% Basic Document Settings
%

\topmargin=-0.45in
\evensidemargin=0in
\oddsidemargin=0in
\textwidth=6.5in
\textheight=9.0in
\headsep=0.25in

\linespread{1.1}

\pagestyle{fancy}

%
% Create Problem Sections
%

\newcommand{\enterProblemHeader}[1]{
    \nobreak\extramarks{}{Question \arabic{#1} continued on next page\ldots}\nobreak{}
    \nobreak\extramarks{Question \arabic{#1} (suite)}{Suite de la question \arabic{#1} à la page suivante\ldots}\nobreak{}
}

\newcommand{\exitProblemHeader}[1]{
    \nobreak\extramarks{Question \arabic{#1} (suite)}{Suite de la question \arabic{#1} à la page suivante\ldots}\nobreak{}
    \stepcounter{#1}
    \nobreak\extramarks{Question \arabic{#1}}{}\nobreak{}
}

\setcounter{secnumdepth}{0}
\newcounter{partCounter}
\newcounter{homeworkProblemCounter}
\setcounter{homeworkProblemCounter}{1}
\nobreak\extramarks{Question \arabic{homeworkProblemCounter}}{}\nobreak{}

%
% Homework Problem Environment
%
% This environment takes an optional argument. When given, it will adjust the
% problem counter. This is useful for when the problems given for your
% assignment aren't sequential. See the last 3 problems of this template for an
% example.
%
\newenvironment{homeworkProblem}[1][-1]{
    \ifnum#1>0
        \setcounter{homeworkProblemCounter}{#1}
    \fi
    \section{Tâche \arabic{homeworkProblemCounter}}
    \setcounter{partCounter}{1}
    \enterProblemHeader{homeworkProblemCounter}
}{
    \exitProblemHeader{homeworkProblemCounter}
}

%
% Homework Details
%   - Title
%   - Due date
%   - Class
%   - Section/Time
%   - Instructor
%   - Author
%

\newcommand{\hmwkTitle}{Travail\ pratique\ 4}
\newcommand{\hmwkDueDate}{8 Décembre 2023}
\newcommand{\hmwkClass}{\ \ IFT 3913}
\newcommand{\hmwkClassTime}{}%Section 
\newcommand{\hmwkClassInstructor}{Professeur: Michalis Famelis}
\newcommand{\hmwkAuthorName}{\textbf{Killian Gervais \& Gabriel Hazan}}

%
% Title Page
%

\title{
    \vspace{2in}
    \textmd{\textbf{\hmwkClass:\ \hmwkTitle}}\\
    \normalsize\vspace{0.1in}\small{Pour\ le\ \hmwkDueDate\ à 23:59 }\\
    \vspace{0.1in}\large{\textit{\hmwkClassInstructor\ \hmwkClassTime}}
    \vspace{3in}
}

\author{\hmwkAuthorName}
\date{}

\renewcommand{\part}[1]{\textbf{\large Partie \Alph{partCounter}}\stepcounter{partCounter}\\}

% Probability commands: Expectation, Variance, Covariance, Bias
\newcommand{\E}{\mathrm{E}}
\newcommand{\Var}{\mathrm{Var}}
\newcommand{\Cov}{\mathrm{Cov}}
\newcommand{\Bias}{\mathrm{Bias}}

\tikzstyle{bag} = [align=center]
\begin{document}

\maketitle
\thispagestyle{empty}

\pagebreak
% Tache 1
\begin{homeworkProblem}
    D'après la spécification donnée, on a deux types d'entrées: les devises (\textit{currencies}) et les montants (\textit{amounts}). Soit les devises du programme $P_C$ défini sur \{USD, CAD, GBP, EUR, CHF, AUD\} avec \\$C$ = Currencies = \{USD, CAD, GBP, EUR, CHF, AUD, JPY, INR, NZD\}, cette dernière pourrait être agrandie au besoin, et les montants du programme $P_A$ défini sur [0, 1 000 000], l'intervalle des valeurs valides, avec $A$ = Amounts = Réels.\\
    \linebreak
    Il y a deux classes d'équivalences pour les devises:
    \begin{itemize}
        \item[$\bullet$] Les valeurs d'entrées valides:\quad $C_1 = P_C =$ \{USD, CAD, GBP, EUR, CHF, AUD\}
        \item[$\bullet$] Les valeurs d'entrées invalides:\quad $C_2 =$ \{JPY, IRN, NZD\}
    \end{itemize}
    On choisit une valeur de chaque classe, pas besoin de plus que cela puique l'échelle est nominale, pour créer notre jeu de test $T_C$ = \{USD, JPY\}.\\
    \linebreak
    Pour les montants, il y a trois classes d'équivalences, $a \in \mathbb{R}$:
    \begin{itemize}
        \item[$\bullet$] Les valeurs appartenant à $P_A$:\quad $A_1 = \{0 \leq a \leq 1\ 000\ 000\}$
        \item[$\bullet$] Les valeurs invalides inférieures à $P_A$:\quad $A_2 = \{a < 0\}$
        \item[$\bullet$] Les valeurs invalides supérieure à $P_A$:\quad $A_3 = \{a > 1\ 000\ 000\}$
    \end{itemize}
    On choisit choisit une valeur "typique" dans chaque classe d'équivalence et plusieurs aux bornes de celles-ci pour créer le jeu de test $T_A = \{-2\ 222\ 222, -0.09, 0, 500\ 000, 1\ 000\ 000, 1\ 000\ 000.01, 2\ 222\ 222\}$.\\

    Si une entrée est invalide, on s'attend à ce que le code ne s'arrête pas, il est donc capable de s'adapter à une mauvaise entrée fournie par l'utilisateur, que ce soit pour une devise ou un montant.
\end{homeworkProblem}

%Tache 2
\begin{homeworkProblem}
    
\end{homeworkProblem}

%Tache 3
\begin{homeworkProblem}
    
\end{homeworkProblem}
\end{document}